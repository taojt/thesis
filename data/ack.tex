% !Mode:: "TeX:UTF-8"
%%% Local Variables:
%%% mode: latex
%%% TeX-master: "../main"
%%% End:

\begin{ack}
  快乐的时间总是过得很快,转瞬之间,我在北京师范大学三年的硕士研究生生活即将结束。在临近毕业之际,我要向曾经帮助和关心我的老师、家人、同学与好友们表示感谢。

  首先感谢我的两位指导老师:余先川教授和胡丹副教授。还记得三年前刚来北京师范大学学习的时候,我对实验室研究工作和科研方向都比较迷茫,是胡丹老师指导我开展科研学习的工作,是胡老师告诉我如何去看论文,怎样学习研究学者思考问题的方式。每周一到两次的讨论中胡老师也时常跟我探讨课题中的知识,对于我的困惑与科研中遇到的问题,胡老师都耐心的和我沟通、讨论。研二下学期伊始,胡丹老师去美国从事新的事业,虽然远隔重洋,但还时常接收到胡老师来自大洋彼岸的关心和问候。同时胡老师在生活上对我也非常热心,有时候生活上不如意或遇到烦心事,胡老师也体贴入微,关心并疏导我的个人问题,感谢胡老师近两年对我学习和生活上的关怀。胡老师出国后,我转到现在的导师余先川教授下继续从事科研学习。我也由衷地感谢余老师,余老师在专业学科领域有极其敏锐的洞察力,能够迅速了解当前领域内研究的前沿内容。在余老师的指导下,我能够在遥感影像数据挖掘领域继续深入研究,提升自己的科研能力。毕业论文从开题到撰写过程中,余老师都认真、耐心地指导我并指出论文中存在的问题与不足。在生活上,余老师待人亲切,关心学生,更像是实验室大家庭的家长,实验室经常性的聚餐、实验室每个小伙伴生日会的庆祝与祝福都让远离家乡,漂泊北京的我们感受到了来自家庭的温暖。此外,余老师也会带着我们去参加一些前沿学术论坛和会议,让我们接触到学术界的科研大牛的同时,也开阔拓展了我们的视野。再次对我两位导师表示致谢!

  其次感谢实验室的张立保老师,张老师严谨的科研态度值得我学习,另外感谢张老师在电子楼512室给我提供的科研工位,让我更便利地从事科研学习。同时感谢师母的关照,感谢师母对实验室学生的关心和支持,师母也让我们体会到慈爱家长般的温暖。

  感谢实验室的每一位同学,和大家在一起学习、生活的时光让我快乐、开心,感谢实验室里每一位师兄师姐师弟和师妹们。

  感谢我的室友冯思博、朱云宗和戎博杰,谢谢你们三年的陪伴,谢谢你们容忍我生活上的一些小毛病,感谢大家。

  最后,我要感谢我的父母、家人和朋友们对我的支持和鼓励,是你们在我迷茫的时候给我支持和力量,鼓舞着我面对困难,笑对困难与挫折。
\end{ack}
