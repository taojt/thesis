% !Mode:: "TeX:UTF-8"
%%% Local Variables:
%%% mode: latex
%%% TeX-master: t
%%% End:

\chapter{遥感影像分割基础}
\label{cha:chap02}

\section{基于无监督聚类的遥感影像分割基础}
\label{sec:other}


本模板不再预先装载任何绘图包(如 \textsf{pstricks,pgf} 等),完全由你自己来决定。
个人觉得 \textsf{pgf} 不错,不依赖于 Postscript。 此外还有很多针对 \LaTeX{} 的
GUI 作图工具,如 XFig(jFig), WinFig, Tpx, Ipe, Dia, Inkscape, LaTeXPiX,
jPicEdt, jaxdraw 等等。

\subsection{图一}
\label{sec:graphs}

强烈推荐《\LaTeXe 插图指南》!关于子图形的使用细节请参看 \textsf{subfig} 的说明文档。

\subsubsection{一个图形}
\label{sec:onefig}
一般图形都是处在浮动环境中。之所以称为浮动是指最终排版效果图形的位置不一定与源文
件中的位置对应\footnote{This is not a bug, but a feature of \LaTeX!},这也是刚使
用 \LaTeX{} 同学可能遇宏包,
%它提供了 \texttt{[H]} 参数。比如图~\ref{fig:xfig1}。
%\begin{figure}[H] % use float package if you want it here
%  \centering
%  \includegraphics{hello}
%  \caption{利用 Xfig 制图}
%  \label{fig:xfig1}
%\end{figure}

\section{基于神经网络的遥感影像分割方法}
\label{sec:chap02-2}

一般图形都是处在浮动环境中。之所以称为浮动是指最终


