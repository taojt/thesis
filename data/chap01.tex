% !Mode:: "TeX:UTF-8"
%%% Local Variables:
%%% mode: latex
%%% TeX-master: t
%%% End:

\chapter{绪论}
\label{cha:chap01}

这是 \bnuthesis{} 的示例文档,基本上覆盖了模板中所有格式的设置。建议大家在使用模
板之前,除了阅读《\bnuthesis{}用户手册》,这个示例文档也最好能看一看。

小老鼠偷吃热凉粉;短长虫环绕矮高粱。\footnote{韩愈(768-824),字退之,}


\section{研究背景及意义}
\label{sec:first}

遥感技术是从各种传感器上收集地物目标的电磁辐射信息,经处理后成像,从而对地物进行探测和识别的一种技术。遥感影像数据被广泛应用在军事侦察、环境监测、植被分类、土地利用规划和矿产资源勘测等领域\cite{lishihua2005}。近年来,随着卫星遥感技术的发展和信息科技技术的完善,遥感影像分辨率不断提高,高分辨率影像信息量越来越丰富。 同时全球遥感数据成爆发式增长,但相关统计表明,遥感数据 $95\%$ 是不精确的、非结构化的数据,人类能够利用的数据仅占 $5\%$ 左右\cite{zhangjun2010},如何在有限时间内高效利用遥感数据是当前遥感技术发展所面临的挑战。

影像分类与目标识别是遥感影像分析和应用的重要内容,如何准确、快速地对遥感影像分类与识别是当前遥感应用领域的研究热点。传统的遥感影像分类方法从人工目视解译发展到人机交互解译,再到半自动解译,最后到当前基于机器学习模型和人工智能技术的全自动解译发展过程;影像分类模型则由传统的像元解译、局部结构特征提取发展到了面向对象识别的阶段;分类器也从单一的分类器发展为层叠或多个分类器相结合的方法\cite{lideren2012}。基于新兴理论提出的新技术、新方法在遥感影像分类与识别研究中取得了较好的识别效果,提升了影像识别的精度。然而,由于遥感影像存在混合像元,同物异谱和同谱异物等问题\cite{wulun2006},遥感影像数据固有的不确定性成为影像分类亟需解决的问题,如果能构建适当模型描述影像地物数据,进而提取目标地物特征信息,这将成为影像分类与目标识别的新思路\cite{he2005comparison}。 同时,遥感影像数据普遍存在样本少、数据分布不均衡等特点,获取有标签的影像数据是昂贵、耗时的,需要极大的时间和人力成本,研究基于少样本的半自动或全自动的影像分类与目标识别方法有着重要的意义。

模糊聚类的优势。。。

深度学习优势。。

本文将从刻画遥感影像数据的不确定性和少样本数据分类两个角度,对高分影像数据进行分析与处理,分别提出新的面向对象的区间二型模糊聚类方法用于遥感影像聚类和基于生成对抗网络的弱监督学习方法用于影像分类与识别, 将影像数据分类与识别结果与验证集 ground-truth 图进行比对,验证本文提出的两种方法在遥感影像分类与识别中的有效性。此外,本文综合提出的两种方法,形成一个完整的处理流程,实现高分影像数据的信息提取与分类识别。



\section{国内外研究现状}
\label{sec:second}
为了方便介绍,本文中将深度学习方法之前的遥感影像分类方法称为传统的遥感影像分类方法。本节内容主要介绍了传统的高分辨率遥感影像分类识别方法和基于深度学习技术\citep{hinton2006fast, bengio2009learning, NIPS2012_4824} 的遥感影像识别分类方法的研究进展和现状。
% ,遥感影像的聚类分割

\subsection{传统的高分辨率遥感影像分类与识别方法}
\label{subsec:1-2-1}
早在1957年,卫星遥感技术就应用到遥感影像分类与识别任务中。目标地物的分类与识别一直以来都是遥感影像分析中的一个基础任务,对于研究目标物体或现象的发展过程与分布规律有着重要意义\cite{jensen1987introductory}。遥感影像分类方法依据是否使用地物类别先验知识分为监督分类和非监督分类。监督分类是指利用样本已有先验类别训练分类模型,模型能够建立样本特征到类别标签的决策映射规则; 非监督分类是指在缺乏样本类别先验知识的前提下,只根据样本数据本身特性进行分类,根据样本相似度划分类别,如聚类\cite{djukanovic1993unsupervised}。根据分类单元不同,遥感影像的分类方法可划分为基于像元和面向对象的分类。基于像元的分类方法以像元的光谱信息作为主要依据进行分类,常见的基于像元的遥感影像分类方法有:最小距离分析法\cite{wacker1972minimum},最大似然分类法\cite{strahler1980use},K-均值聚类法\cite{atkinson2000geostatistical} 和 ISODATA 聚类法\cite{paul2002new}等。随着遥感技术不断发展与成熟,遥感影像空间分辨率不断提高。一般地,我们将空间分辨率高于 $5m$ 遥感影像称作高分辨率遥感影像\cite{zhangyongsheng2004}。高分影像相比低分辨率影像来说光谱信息相对匮乏,而高分影像的几何、纹理等信息却更加丰富。基于像元的分类方法应用到高分辨率影像中会导致影像解译速度慢,同时椒盐噪声极易产生,因而其不适用于高分影像分类\cite{blaschke2010object}。

面向对象的高分辨率影像分类方法将影像中邻域同质像元组成的对象作为分类单元,充分利用影像地物的形状、纹理等特征, 更适合高分影像分类与识别\cite{zhangyongsheng2004}。 早在1976年,Kettig 和 Robert \cite{kettig1976classification} 就将面向对象的思想引入遥感影像研究领域中。随后,Lobo 等人 \cite{lobo1996classification} 将面向对象分类方法应用到遥感影像分类中,通过实验证明了在高分影像识别任务中面向对象的分类方法比基于像元的方法识别速度更快,分类精度更高。Baatz \cite{baatz1999object} 基于高分辨率遥感影像特性,系统地提出高分影像的面向对象分类方法。之后,面向对象分类方法被广泛应用到高分影像分类识别任务中,发展迅速。 Geneletti \cite{geneletti2003method} 和 Guo \cite{guo2007object} 分别从非监督分类的研究方向表明面向对象分类方法是基于像元方法的有效替代。在工业应用领域,德国 Definiens 公司于2009 年开发的 Ecognition 影像分析软件极大的推动了面向对象的高分影像分类方法的商业发展,同时也表明了面向对象的高分影像方法的成熟。

一般的, 面向对象的遥感影像分类方法通常包含三个部分: 影像分割,特征提取和分类识别。 高分影像因其空间分辨率高,纹理、形状等空间信息相对丰富,分割方法和精度便成为影像分割的关键要素。Canny 通过提出 Canny 算子 \cite{canny1987computational} 检测出影像所有边缘点,并将边缘点依次连接形成边界从而实现影像边缘分割。Otsu 基于灰度直方图动态计算图像分割中的阈值范围,形成不同目标间差异最大化,实现阈值分割\cite{otsu1979threshold}。 Vincent \cite{vincent1991watersheds} 等人结合沉浸模型提出影像的分水岭分割。 Achanta 和 Radhakrishna \cite{achanta2012slic} 基于K-均值聚类方法,采用简单的迭代聚类高效地生成影像分割单元,提出 SLIC 超像素分割算法,该方法目前被广泛应用到影像分割中。在特征提取阶段,最初采用影像的光谱、纹理和形状等低阶特征信息,但低阶特征无法获得较好的分类效果。文献 \cite{weizman2009urban} 中引入词包模型的中层语义特征实现对遥感影像信息更好的表达,实验结果表明该方法分类效果更好。 随后,Lienou 等人\cite{lienou2010semantic} 将主题模型应用到词包模型的单词语义分析中,改进了前者的分类精度。He 等人 \cite{he2016remote} 针对遥感影像同物异谱的现象,结合模糊数学中不确定性理论的相关方法,设计一种区间值特征来提取目标地物的特征,用于面向对象的非监督分类。目前,在特征提取的方面研究者做了大量工作,然而,高级特征的表达仍需要复杂的人工设计和反复实验验证。 分类识别阶段是针对特征提取阶段获得的分割对象的特征,利用分类器对待识别目标进行分类识别。目前,常用的机器学习分类方法包含随机森林 \cite{pal2005random},支持向量机 \cite{suykens1999least} ,决策树 \cite{friedl1997decision} 和神经网络模型 \cite{haykin1994neural} 等。 在这些分类器基础上,通过结合不同分类器延申而出的集成学习 \cite{freund1996experiments} 的方法也被应用到高分影像分类中。

然而,传统的高分影像的分类与识别方法只应用到影像中、低层特征,无法充分表达影像信息,而采用的影像分类器大多是只有 $1\sim2$ 层的浅层结构模型,无法充分学习和表达遥感影像复杂的数据结构和特征信息。因此,研究结构更复杂,表达能力更强的分类识别模型具有必要的意义。

\subsection{基于深度学习技术影像识别与分类研究现状}
\label{subsec:1-2-2}
深度学习的概念源于人工神经网络,最早由 Geoffrey Hinton \cite{hinton2006fast} 教授于2006年提出。深度学习通过组合低层特征形成更加抽象的高层表示属性类别或特征,从大量数据中自动学习数据的特征表示。深度学习利用多层网络模型学习抽象概念完成自我学习 \cite{lecun2015deep}。 深度学习最早应用于图像处理领域,目前在自然语言处理,语音识别,搜索推荐,游戏AI 和自动驾驶等领域广泛应用,且均表现出卓越的效果 \cite{bengio2009learning}。2012 年,Alex 等人在ILSVRC 图像识别大赛中提出了基于卷积神经网络(Convolutional Neuro Network, CNN) 结构的AlexNet \cite{NIPS2012_4824} 模型,对ImageNet 数据集上千万级的自然图像进行分类,大幅提升了图像分类精度。AlexNet 的提出首次证明了 CNN 在复杂模型下的有效性,并极大推动了有监督深度学习领域的发展。在2014年 ILSVRC 大赛上,基于 CNN 结构,Google 研究团队提出的 GoogLeNet \cite{szegedy2015going} 和牛津大学学者提出的 VGGNet \cite{simonyan2014very} 分别荣获当年 ImageNet 识别大赛的一、二名。 这两者在 AlexNet 的基础上均探索了网络深度与性能的关系,实验结果也证明了增加网络深度在一定程度上会影响网络最终的性能,使得分类错误率大幅下降。另外,GoogLeNet 中提出的 Inception 结构和 VGGNet 中提出的小卷积核多层网络结构也大幅优化了网络参数的数量,提升了训练学习速度的同时使得网络分类效果更优秀,且具有优秀的扩展泛化能力。之后,何凯明在 ResNet \cite{he2016deep} 网络模型中创造性地提出了残差学习(Residual Learning)的概念,解决了深度学习随着网络层数加深网络退化问题,使得更深层次网络模型得以训练,同时 ResNet 一并刷新了当年 ILSVRC 和 COCO 2015 图像识别大赛的最优记录。在非监督学习领域,深度学习算法模型近十年也发展迅速。2006年, Hinton 对传统自动编码器结构进行改进,提出了深度自编码网络(Deep AutoEncoder, DAE) \cite{hinton2006fast}。DAE 网络利用无监督逐层贪心训练算法完成对隐含层的预训练,然后用 BP 算法对整个网络参数进行调整,显著降低了深层自编码结构的性能指数,且大幅提升自编码器的学习能力。之后,基于 DAE 理论相继提出的栈式自动编码器(stacked AutoEncoder, stacked AE)\cite{bengio2007greedy}、降噪编码器(Denoise Autoencoder, dAE)\cite{vincent2008extracting} 和稀疏自编码器(Sparse AutoEncoder, SAE)\cite{ng2011sparse} 等均取得了不错的效果。2014年,Goodfellow 结合二人零和博弈的思想,创造性地提出了生成对抗网络(Generative Adversarial Net,GAN)\cite{goodfellow2014generative} 模型, 极大地促进了无监督学习和计算机视觉领域(如图片生成、风格迁移和图像分割等)的发展。GAN 模型框架由两个“对抗”模型组成:捕获数据分布的生成模型 G 和估计样本来自训练数据而不是 G 的概率的判别模型 D 。

\section{本文的组织结构}
\label{sec:third}
封面的例子请参看 cover.tex。主要符号表参看 denation.tex,附录和

\section{本文主要创新点}
\label{sec:forth}
封面的例子请参看 cover.tex。主要符号表参看 denation.tex,附录和
