% !Mode:: "TeX:UTF-8"
%%% Local Variables:
%%% mode: latex
%%% TeX-master: t
%%% End:

\chapter{绪论}
\label{cha:chap01}

这是 \bnuthesis{} 的示例文档,基本上覆盖了模板中所有格式的设置。建议大家在使用模
板之前,除了阅读《\bnuthesis{}用户手册》,这个示例文档也最好能看一看。

小老鼠偷吃热凉粉;短长虫环绕矮高粱。\footnote{韩愈(768-824),字退之,}


\section{研究背景及意义}
\label{sec:first}

遥感技术是从各种传感器上收集地物目标的电磁辐射信息,经处理后成像,从而对地物进行探测和识别的一种技术。遥感影像数据被广泛应用在军事侦察、环境监测、植被分类、土地利用规划和矿产资源勘测等领域\cite{lishihua2005}。近年来,随着卫星遥感技术的发展和信息科技技术的完善,遥感影像分辨率不断提高,高分辨率影像信息量越来越丰富。 同时全球遥感数据成爆发式增长,但相关统计表明,遥感数据 $95\%$ 是不精确的、非结构化的数据,人类能够利用的数据仅占 $5\%$ 左右\cite{zhangjun2010},如何在有限时间内高效利用遥感数据是当前遥感技术发展所面临的挑战。

影像分类与目标识别是遥感影像分析和应用的重要内容,如何准确、快速地对遥感影像分类与识别是当前遥感应用领域的研究热点。传统的遥感影像分类方法从人工目视解译发展到人机交互解译,再到半自动解译,最后到当前基于机器学习模型和人工智能技术的全自动解译发展过程;影像分类模型则由传统的像元解译、局部结构特征提取发展到了面向对象识别的阶段;分类器也从单一的分类器发展为层叠或多个分类器相结合的方法\cite{lideren2012}。基于新兴理论提出的新技术、新方法在遥感影像分类与识别研究中取得了较好的识别效果,提升了影像识别的精度。然而,由于遥感影像存在混合像元,同物异谱和同谱异物等问题\cite{wulun2006},遥感影像数据固有的不确定性成为影像分类亟需解决的问题,如果能构建适当模型描述影像地物数据,进而提取目标地物特征信息,这将成为影像分类与目标识别的新思路\cite{he2005comparison}。 同时,遥感影像数据普遍存在样本少、数据分布不均衡等特点,获取有标签的影像数据是昂贵、耗时的,需要极大的时间和人力成本,研究基于少样本的半自动或全自动的影像分类与目标识别方法有着重要的意义。

本文将从刻画遥感影像数据的不确定性和少样本数据分类两个角度,对高分辨率遥感影像数据进行分析与处理,分别提出新的面向对象的区间二型模糊聚类方法用于遥感影像聚类和基于生成对抗网络的弱监督学习方法用于影像分类与识别, 将影像数据分类与识别结果与验证集 ground-truth 图进行比对,验证本文提出的两种方法在遥感影像分类与识别中的有效性。此外,本文综合提出的两种方法,形成一个完整的处理流程,实现高分辨率影像数据的信息提取与分类识别。



\section{国内外研究现状}
\label{sec:second}
本节内容主要介绍了高分辨率遥感影像分类识别、遥感影像的聚类分割和基于深度学习技术\citep{hinton2006fast, bengio2009learning, NIPS2012_4824} 图像识别分类的研究进展和现状。

\subsection{高分辨率遥感影像分类现状}
\label{subsec:1-2-1}
目标地物的分类与识别一直以来都是遥感影像解译


\subsection{遥感影像的聚类分割现状}
\label{subsec:1-2-2}
ceshier

\subsection{基于深度学习技术图像识别分类的研究现状}
\label{subsec:1-2-3}
顶顶顶

\section{本文的组织结构}
\label{sec:third}
封面的例子请参看 cover.tex。主要符号表参看 denation.tex,附录和

\section{本文主要创新点}
\label{sec:forth}
封面的例子请参看 cover.tex。主要符号表参看 denation.tex,附录和
