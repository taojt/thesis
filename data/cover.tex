% !Mode:: "TeX:UTF-8"
%%% Local Variables:
%%% mode: latex
%%% TeX-master: t
%%% End:
%\secretlevel{绝密} \secretyear{10}

\ctitle{基于生成对抗网络的遥感影像分类研究}
%\makeatletter
%\cdegree{硕士}

\makeatother

\cauthor{江涛}
\csupervisor{余先川教授、胡丹副教授} %\\ \hspace*{6em}}
\cdepartment[]{信息科学与技术学院16级}
\cmajor{计算机软件与理论}
\cnum{201621210026}
\cdate{\the\year 年 \the\month 月}

\etitle{Remote Sensing Image Classification Based on Fuzzy Clustering and Generative Adversarial Network}

% 定义中英文摘要和关键字
\begin{cabstract}
  遥感影像分类是遥感领域重要的研究内容,快速、准确地对遥感影像数据分类与识别是当前遥感领域的研究重点。然而,由于遥感数据固有的不确定性使得影像准确解译识别面临挑战。此外,高分辨率影像在提供高质量细节信息时其同类地物类内特征差异大,使得分类容易错分类内差异大的地物,且难以预测出明确、清晰的地物分类边界。

  近年来,随着深度学习技术的发展和全卷积分类网络的出现,基于深度学习理论的影像分类方法被广泛应用到遥感影像分类任务中。但目前基于全卷积网络的影像分类方法在上采样操作中易使影像丢失特征信息,导致地物分类边界模糊,另一方面,全卷积影像分类方法是像素级的分类,影像分割结果中邻近区域同类地物预测标签难以保证一致性。因此,改进现有的全卷积影像分类方法,提升影像分类质量,也具有重要的研究意义。生成对抗网络其判别器能够不断检测并纠正真实样布与生成数据间的差异,保持整体的一致性,且对抗博弈过程中生成器能不断强化分割模型的预测能力。因此,将对抗训练的思想引入全卷积影像分类方法有助于提升分类精度。此外,利用模糊聚类方法对遥感影像不确定信息建模,将影像聚类结果作为先验辅助知识添加到全卷积影像分类模型中,能够增强分类模型对遥感影像不确定性信息的表征能力,提升分类精度。基于上述思路,本课题论文对遥感影像地物分类问题加以研究和讨论。具体的研究内容如下:
  \begin{enumerate}[(1)]
    \item 将对抗训练思想引入全卷积影像分类模型,提出基于条件对抗生成网络的全卷积影像分类方法,新方法在高分影像数据上相比原方法分类能力更强,同类别地物空间更加具有一致性。
    \item 利用作者先前提出的面向对象的遥感影像模糊聚类方法处理影像数据,模糊聚类输出结果作为影像区域领域关系和同质性的特征图,将特征图引入提出的基于条件对抗生成网络的影像分类方法中,联合训练,提升影像分类结果同类别地物一致性,并得到更准确的地物分类边界。
  \end{enumerate}
\end{cabstract}

\ckeywords{遥感影像, 地物分类, 全卷积网络, 生成对抗网络,深度学习 }

\begin{eabstract}
  Remote sensing image classification is a research hotspot in the field of remote sensing. How to quickly and accurately classify and identify  of remote sensing images is the focus of current remote sensing research. However, it is challenging to accurately classify remote sensing images, due to the inherent uncertainty of remote sensing images. High-resolution images provide high-quality detail information with large differences in their similar species, which leads to the misclassification, and it is difficult to get the correct classification boundary.

    Object-oriented fuzzy clustering segmentation methods are widely used to handle uncertainty in high-resolution images because of its excellent ability to characterize uncertain information. In the current object-oriented segmentation methods, the image segmentation unit adopts mean data modeling or interval value data modeling. However, accurately distinguishing
  two objects with the same mean or interval values and different internal distributions is difficult. Therefore, we designed a triangular fuzzy set modeling to describe objects and designed an interval distance metric to measure the dissimilarities between triangular fuzzy sets. The object-oriented clustering methods belong to unsupervised classification methods, which required manual post-processing. With the development of deep learning and the emergence of a full convolutional network, supervised image segmentation methods are also widely used in remote sensing image classification. In full convolutional network, the upsampling operation causes the loss of image features, which leads to the blurring of the boundary of the feature classification. Designing a new network model to improve the spatial consistency of image classification of similar features also has research significance. Based on the above two ideas, this paper will analyze and discuss the remote sensing image feature classification from the perspective of image segmentation unit data modeling and ground object consistency processing. The specific research contents are as follows:
\begin{enumerate}[(1)]
  \item Improves the existing object-oriented fuzzy clustering segmentation method from the perspective of image segmentation unit feature modeling. In this paper, a triangular fuzzy set modeling was designed to describe objects, and an interval distance metric was proposed to measure the dissimilarities between triangular fuzzy sets. Finally, the triangular fuzzy set-valued data interval type 2  fuzzy c means clustering method was proposed. The clustering results of SPOT-5 and Gaofen-2 high-resolution image data verified that the proposed method results in improved classification quality and accuracy.
  \item Improves the existing full convolutional supervised classification method from the perspective of feature consistency. This paper draws on the excellent image generation ability of the generative adversarial network, applies the conditional generative adversarial network to the existing full convolutional network segmentation model, and proposes a full convolutional image segmentation method based on conditional generative adversarial network. The segmentation results of the proposed method on the Vaihingen image dataset have more accurate feature boundaries and consistency with the same type of feature space.
  
\end{enumerate}

\end{eabstract}

\ekeywords{Remote Sensing Imagery, Land Cover Classification, Fully Convolutional Network, Generative Adversarial Network, Deep Learning}
