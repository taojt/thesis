% !Mode:: "TeX:UTF-8"
%%% Local Variables:
%%% mode: latex
%%% TeX-master: t
%%% End:
%\secretlevel{绝密} \secretyear{10}

\ctitle{基于模糊聚类及生成对抗网络的遥感影像分类方法研究}
%\makeatletter
%\cdegree{硕士}

\makeatother

\cauthor{江涛}
\csupervisor{余先川教授} %\\ \hspace*{6em}}
\cdepartment{信息科学与技术学院}
\cmajor{计算机软件与理论}
\cnum{201621210026}
\cdate{\the\year 年 \the\month 月}

\etitle{Remote Sensing Image Classification Based on Fuzzy Clustering and Generative Adversarial Network}

% 定义中英文摘要和关键字
\begin{cabstract}
  遥感影像分类是遥感领域重要的研究内容,快速、准确地对遥感影像数据分类与识别是遥感领域当前地研究重点。然而,由于遥感数据固有的不确定性使得影像准确解译识别面临挑战。
\end{cabstract}

\ckeywords{遥感影像, 地物分类, 模糊聚类, 神经网络, 全卷积网络, 生成对抗网络 }

\begin{eabstract}
   An abstract of a dissertation is a summary and extraction of research work
   and contributions. Included in an abstract should be description of research
   topic and research objective, brief introduction to methodology and research
   process, and summarization of conclusion and contributions of the
   research. An abstract should be characterized by independence and clarity and
   carry identical information with the dissertation. It should be such that the
   general idea and major contributions of the dissertation are conveyed without
   reading the dissertation.

   An abstract should be concise and to the point. It is a misunderstanding to
   make an abstract an outline of the dissertation and words ``the first
   chapter'', ``the second chapter'' and the like should be avoided in the
   abstract.

   Key words are terms used in a dissertation for indexing, reflecting core
   information of the dissertation. An abstract may contain a maximum of 5 key
   words, with semi-colons used in between to separate one another.
\end{eabstract}

\ekeywords{Remote Sensing Images, Land Cover Classification, Fuzzy Clustring, Convolutional Neural Network, Fully Convolutional Network, Generative Adversarial Network}
